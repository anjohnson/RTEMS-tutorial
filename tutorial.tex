\documentclass{report}

\usepackage{epsfig}
\usepackage{times}
\usepackage{html}
\usepackage{alltt}
\usepackage{verbatimfiles}
\usepackage{ifthen}

\textwidth=6.5in
\topmargin=0pt
\headheight=0pt
\textheight=8.6truein
\oddsidemargin=0in
\evensidemargin=0in
\footskip=40pt

\parindent=0pt
\parskip=0.5ex

%
% Versions
%
\input versions.tex
\newcommand{\rtemsVersion}{4.7}
\newcommand{\rtemsPackages}{20041017}

%
% Where to get the GNU tools
%
\newcommand{\RTEMSBINUTILSURL}{\RTEMSSOURCEURL/\BINUTILS.tar.bz2}
\newcommand{\RTEMSGCCURL}{\RTEMSSOURCEURL/\GCC.tar.bz2}
\newcommand{\RTEMSNEWLIBURL}{\RTEMSSOURCEURL/\NEWLIB.tar.gz}
\newcommand{\RTEMSBINUTILSDIFFURL}{\RTEMSSOURCEURL/\BINUTILS-rtems-\BINUTILSDIFF.diff}
\newcommand{\RTEMSGCCDIFFURL}{\RTEMSSOURCEURL/\GCC-rtems-\GCCDIFF.diff}
\newcommand{\RTEMSNEWLIBDIFFURL}{\RTEMSSOURCEURL/\NEWLIB-rtems-\NEWLIBDIFF.diff}

\newcommand{\Bar}[1]{$\overline{\mbox{#1}}$}
\newcommand{\RWbar}{$\mbox{R}/\overline{\mbox{W}}$}

\title{Getting started with EPICS on RTEMS}
\author{W. Eric Norum}

\begin{document}
\maketitle
\newpage
\pagenumbering{roman}
\tableofcontents
\newpage
\pagenumbering{arabic}

\chapter{Introduction}
This tutorial presents the steps needed to obtain and install
the development tools and
libraries required  to run EPICS IOC applications using RTEMS.
As you can see by the size of this document the process isn't trivial,
but it's not terribly difficult either.

Chapter two deals with the problem of getting all the tools in place.  This
is the most difficult task.  Once the tools and operating system are working
most of the work is complete.  Chapter three shows the steps needed to
configure and build your first EPICS application for RTEMS.  After you've
completed those steps you can forget about this document and use the
generic EPICS documentation.

\begin{htmlonly}
If you'd like to print this tutorial you can download a
\htmladdnormallink{PDF version}{tutorial.pdf}.
\end{htmlonly}


This is a living document.  Please let me
(\htmladdnormallink{norume@aps.anl.gov}{mailto:norume@aps.anl.gov})
know which of these instructions worked for you and which did not.


\chapter{Infrastructure -- Tools and Operating System}
\section{Create the RTEMS source and installation directories}
There should be at least 300 Mbytes of space available on the drive
where these directories are located.
The suggested location for the installation directory is \verb@/opt/rtems/rtems\rtemsToolVersion@.
If you must place the root of the RTEMS installation in some other
location, it is still a good idea to make a symbolic link from \verb@/opt/rtems/rtems\rtemsToolVersion@
to the actual root of the RTEMS installation tree.
The location of the RTEMS source is not critical.  This document assumes
that the root of the RTEMs source tree is \verb@/usr/local/source/RTEMS@.

Create the directories where the source will be placed
and the results of the build installed:
\begin{verbatim}
/usr/local/source/RTEMS
/usr/local/source/RTEMS/tools
/opt/rtems/rtems\rtemsToolVersion
\end{verbatim}

\section{Add the directory containing the tools to your shell search path}
The following sections assume that the directory into which
you will install the cross-development tools
(\verb@/opt/rtems/rtems\rtemsToolVersion/bin@) is on your shell search path.
For shells
like sh, bash, zsh and ksh you can to this with
\begin{verbatim}
PATH="$PATH:/opt/rtems/rtems\rtemsToolVersion/bin"
\end{verbatim}
For shells like csh and tcsh you can
\begin{verbatim}
set path = ( $path /opt/rtems/rtems\rtemsToolVersion/bin )
\end{verbatim}


\section{Get and build the development tools}
RTEMS uses the GNU toolchain to build the executive and libraries.
Information about the GNU tools can be found
on the \htmladdnormallink{GNU home page}{http://www.gnu.org}.
If you're feeling brave you can skip the following sections and turn loose
the script included in appendix\ref{getAndBuildTools}.
\begin{htmlonly}
The script can also be \htmladdnormallink{downloaded from here}{getAndBuildTools.sh}.
\end{htmlonly}
The script
attempts to download, unpack, configure, build and install
the GNU cross-development
tools and libraries for one or more target architectures.  To use
the script, set the \verb@ARCHS@ 
environment variable to the architectures you wish to support, then 
\begin{verbatim}
sh getAndBuildTools.sh
\end{verbatim}
Set the MAKE environment variable to the name of whatever make program you
need for your system.

\subsection {Download the tool source files}
The source for the GNU tools should be obtained from
the On-line Applications Research (OAR) FTP server since
that server provides any RTEMS-specific patches that may have to be
applied before the tools can be built.

The files in the OAR FTP server directory
{\small \RTEMSSOURCEURL}
should be downloaded to the RTEMS/tools directory created above.
The files can be downloaded using a web browser or a command-line program such as \verb@curl@ or \verb@wget@. (note that the
command examples have been split to help them fit on the page):
{\small
\begin{alltt}
curl --remote-name
\hspace{0.2in} \RTEMSBINUTILSURL
curl --remote-name
\hspace{0.2in} \RTEMSGCCURL
curl --remote-name
\hspace{0.2in} \RTEMSNEWLIBURL
\ifthenelse{\equal{}{\BINUTILSDIFF}}{}{curl --remote-name
\hspace{0.2in} \RTEMSBINUTILSDIFFURL}
\ifthenelse{\equal{}{\GCCDIFF}}{}{curl --remote-name
\hspace{0.2in} \RTEMSGCCDIFFURL}
\ifthenelse{\equal{}{\NEWLIBDIFF}}{}{curl --remote-name
\hspace{0.2in} \RTEMSNEWLIBDIFFURL}
\end{alltt}
or
\begin{alltt}
\verb@wget --passive-ftp --no-directories --retr-symlinks@
\hspace{0.2in} \RTEMSBINUTILSURL
\verb@wget --passive-ftp --no-directories --retr-symlinks@
\hspace{0.2in} \RTEMSGCCURL
\end{alltt}
...
}

Depending on the type of firewall between your machine and the OAR FTP server
you may need to remove the \verb@--passive-ftp@ option from the {\tt wget} commands.


\subsection {Unpack the source archives:}

The following commands will extract the GNU tool sources from the
downloaded tar archive files.
\begin{alltt}
bzcat \BINUTILS.tar.bz2 | tar xf -
bzcat \GCC.tar.bz2 | tar xf -
zcat \NEWLIB.tar.gz | tar xf -
\end{alltt}

To build the newlib libraries needed by RTEMS you must make a symbolic
link to the newlib source directory from the gcc source directory.
\begin{alltt}
cd \GCC
rm -rf newlib
ln -s ../\NEWLIB/newlib newlib
cd ..
\end{alltt}

\subsection {Apply any RTEMS-specific patches}
If any patch files (those with a \verb@.diff@ suffix) were downloaded
from the OAR FTP server the patches in those files must be applied before
the tools can be compiled.

\ifthenelse{\equal{}{\BINUTILSDIFF}}{}{
Here is how the patches can be applied to the binutils sources:
\begin{alltt}
cd \BINUTILS \newline
patch -p1 <../\BINUTILS-rtems-\BINUTILSDIFF.diff \newline
cd .. \newline
\end{alltt}
}

\ifthenelse{\equal{}{\GCCDIFF}}{}{
Here is how the patches can be applied to the gcc sources:
\begin{alltt}
cd \GCC \newline
patch -p1 <../\GCC-rtems-\GCCDIFF.diff \newline
cd .. \newline
\end{alltt}
}

\ifthenelse{\equal{}{\NEWLIBDIFF}}{}{
Here is how the patches can be applied to the newlib sources:
\begin{alltt}
cd \NEWLIB \newline
patch -p1 <../\NEWLIB-rtems-\NEWLIBDIFF.diff \newline
cd .. \newline
\end{alltt}
}





\subsection {Configure, build and install the `binutils':}
The commands in this section must be repeated for each desired target
architecture.  The examples shown build the tools for Motorola Power PC targets.

\begin{enumerate}
\item Create a directory in which the tools will be built and
change to that directory.
\begin{verbatim}
rm -rf build
mkdir build
cd build
\end{verbatim}

\item Configure the tools.
\begin{alltt}
../\BINUTILS/configure --target=powerpc-rtems\rtemsToolVersion --prefix=/opt/rtems/rtems\rtemsToolVersion
\end{alltt}
You should replace the `\verb@powerpc@' with the name of the architecture
for which you're
building the tools.  Common alternatives are `\verb@m68k@' and `\verb@i386@' for
the Motorola M68k and Intel x86 family of processors, respectively.

\item Build and install the tools.
\begin{verbatim}
make -w all install
\end{verbatim}
In this and all subsequent cases the use of a GNU make program is required.
On some hosts you'll have to use \verb@gmake@ instead of \verb@make@.

\item Return to the directory containing the tool and library sources.
\begin{verbatim}
cd ..
\end{verbatim}
\end{enumerate}

\subsection {Configure, build and install the cross-compiler and libraries}

\begin{enumerate}
\item Create a directory in which the tools will be built and
change to that directory.
\begin{verbatim}
rm -rf build
mkdir build
cd build
\end{verbatim}

\item Configure the compiler and libraries.
\begin{alltt}
../\GCC/configure --target=powerpc-rtems\rtemsToolVersion --prefix=/opt/rtems/rtems\rtemsToolVersion \verb@\@
            --with-gnu-as --with-gnu-ld --with-newlib --verbose \verb@\@
            --with-system-zlib --disable-nls \verb@\@
            --enable-version-specific-runtime-libs \verb@\@
            --enable-threads=rtems \verb@\@
            --enable-languages=c,c++ 
\end{alltt}
You should again replace the '\verb@powerpc@' with the name of the architecture
for which you're
building the cross-compiler and libraries.

\item Build and install the cross-compiler and libraries by.
\begin{verbatim}
make -w all install
\end{verbatim}

\item Return to the directory containing the tool and library sources.
\begin{verbatim}
cd ..
\end{verbatim}
\end{enumerate}



\section{Get, build and install RTEMS}
\subsection {Download the RTEMS source from the OAR web server.}

At the time of writing no release of RTEMS was capable of
supporting EPICS so you had to, and may still have to, work with a
developer's snapshot.
The latest snapshot is available in

\begin{verbatim}
ftp://www.rtems.com/pub/rtems/snapshots/rtems/current/
\end{verbatim}


The compressed {\it tar} archive in this directory can be downloaded using a web browser or a command-line program such as \verb@curl@ or \verb@wget@:
{\small
\begin{alltt}
\verb@curl --remote-name@
\hspace{0.2in}'ftp://www.rtems.com/pub/rtems/snapshots/rtems/current/*.tar.bz2'
\end{alltt}
or
\begin{alltt}
\verb@wget --passive-ftp --no-directories --retr-symlinks@
\hspace{0.2in}'ftp://www.rtems.com/pub/rtems/snapshots/rtems/current/*.tar.bz2'
\end{alltt}
}
Depending on the type of firewall between your machine and the OAR FTP server
you may need to remove the \verb@--passive-ftp@ option from the {\tt wget} command.


When you are done you should have the compressed archive with a name something like
\begin{alltt}
rtems-\rtemsVersion.tar.bz2
\end{alltt}

\subsection {Unpack the RTEMS sources}
Change to your RTEMS source directory and unpack the RTEMS sources by:
\begin{alltt}
bzcat rtems-\rtemsVersion.tar.bz2 | tar xf -
\end{alltt}
This will create the directory {\tt rtems-\rtemsVersion} 
and unpack all the RTEMS source into that directory.

\subsection {Make changes to the RTEMS source to reflect your local conditions.}
Some of the board-support-packages distributed with RTEMS may require
modifications to match the hardware in use at your site.  The following 
sections describe changes commonly made to two of these board-support-packages.
\subsubsection {MVME167}
The linker script distributed with RTEMS assumes an MVME167 with 4~Mbytes of
on-board memory starting at location 0x00800000.  A more common configuration
is 16~Mbytes of memory starting at location 0x00000000.  To reflect this
configuration make the following changes to
\begin{alltt}
rtems-\rtemsVersion/c/src/lib/libbsp/m68k/mvme167/startup/linkcmds
\end{alltt}
\begin{verbatim}
@@ -24,8 +24,8 @@
 /*
  * Declare some sizes. Heap is sized at whatever ram space is left.
  */
-_RamBase = DEFINED(_RamBase) ? _RamBase : 0x00800000;
-_RamSize = DEFINED(_RamSize) ? _RamSize : 4M;
+_RamBase = DEFINED(_RamBase) ? _RamBase : 0x0;
+_RamSize = DEFINED(_RamSize) ? _RamSize : 16M;
 _HeapSize = DEFINED(_HeapSize) ? _HeapSize : 0;
 _StackSize = DEFINED(_StackSize) ? _StackSize : 0x1000;
 
@@ -35,7 +35,7 @@
       This is where we put one board. The base address should be
       passed as a parameter when building multiprocessor images
       where each board resides at a different address. */
-  ram  : org = 0x00800000, l = 4M
+  ram  : org = 0x00000000, l = 16M
   rom  : org = 0xFF800000, l = 4M
   sram : org = 0xFFE00000, l = 128K
 }
\end{verbatim}

\subsubsection {PC-x86}
A change I like to make to the RTEMS pc386 source is
to increase the number of lines on the console display from 25 to 50 since
I find that the output from some EPICS commands  scrolls off the display
when only 25 lines are present.
To make this change, add the `\verb@#define@' line shown below
\begin{alltt}
rtems-\rtemsVersion/c/src/lib/libbsp/i386/pc386/start/start16.S
\end{alltt}
\begin{verbatim}
 +------------------------------------------------------*/
 
 #include <bspopts.h>
 #define RTEMS_VIDEO_80x50
 
 /*--------------------------------------------------------+ | Constants
\end{verbatim}

Another change I make is to automatically fall back to using COM2: as a
serial-line console (9600-8N1) if no video adapter is present.  This allows
the pc386 BSP to be used on conventional PCs with video adapters as well
as with embedded PCs (PC-104) which have no video adapters.
To make this change, add the `\verb@#define@' line shown below
\begin{alltt}
rtems-\rtemsVersion/c/src/lib/libbsp/i386/pc386/console/console.c
\end{alltt}
\begin{verbatim}
   */
  rtems_termios_initialize ();

#define RTEMS_RUNTIME_CONSOLE_SELECT
#ifdef RTEMS_RUNTIME_CONSOLE_SELECT
  /*
   * If no video card, fall back to serial port console
\end{verbatim}

\subsection{Build and install RTEMS}
\label{RTEMS_BSP_CONFIG}
\begin{enumerate}
\item
It is best to start with a clean slate.  {\bf Make very sure you're in the
right directory before running the rm command!}
\begin{verbatim}
cd /usr/local/source/RTEMS
rm -rf build
mkdir build
cd build
\end{verbatim}

\item
Configure RTEMS for your target architecture:
\begin{alltt}
cd /usr/local/source/RTEMS/build
../rtems-\rtemsVersion/configure \verb@--target=powerpc-rtems\rtemsToolVersion --prefix=/opt/rtems/rtems\rtemsToolVersion \@
  \verb@  --enable-cxx --enable-rdbg --disable-tests --enable-networking \@
  \verb@  --enable-posix --enable-rtemsbsp=mvme2100 \@
\end{alltt}
You should replace the `\verb@powerpc@' with the name of the architecture
for which you're building RTEMS.  Common alternatives
are `\verb@m68k@' and `\verb@i386@' for
the Motorola M68k and Intel x86 family of processors, respectively.
Your should replace the `\verb@mvme2100@' with the board-support packages
for your particular hardware

You can build for more than one board-support package by specifying
more names on the command line.  For example,
you could build for a generic MC68360 system and an MVME-167 system
by:
\begin{alltt}
cd /usr/local/source/RTEMS/build
../rtems-\rtemsVersion/configure \verb@--target=m68k-rtems\rtemsToolVersion --prefix=/opt/rtems/rtems\rtemsToolVersion \@
  \verb@  --enable-cxx --enable-rdbg --disable-tests --enable-networking \@
  \verb@  --enable-posix --enable-rtemsbsp="gen68360 mvme167" \@
\end{alltt}

\end{enumerate}


\section{Get, build and install some RTEMS add-on packages}
The EPICS IOC shell uses the the libtecla or GNU readline package to provide command-line editing and command history.
While the IOC shell can be compiled without these capabilities I think they're important enough to warrant
making the effort to download and install the extra packages.
GNU readline is more well-tested, but libtecla does not bring along
the problems associated with the GNU Public License.

EPICS can use either TFTP or NFS for remote file access.
NFS provides considerably more flexibility so it's a good
idea to install this add-on package as well.

\subsection{Download the add-on package sources}
The latest versions of these files are in
\begin{verbatim}
ftp://ftp.rtems.com/pub/rtems/snapshots/contrib/add_on_packages/
\end{verbatim}

The compressed {\it tar} archive in this directory can be downloaded using a web browser or a command-line program such a \verb@wget@ (note that the
wget command example has been split to make it fit on this page):
{\small
\begin{alltt}
\verb@wget --passive-ftp --no-directories --retr-symlinks@
\hspace{0.3in}"ftp://ftp.rtems.com/pub/rtems/snapshots/contrib/add_on_packages/rtems-addon-packages-\rtemsPackages.tar.bz2"
\end{alltt}
}
Depending on the type of firewall between your machine and the OAR FTP server
you may need to remove the \verb@--passive-ftp@ option from the {\tt wget} command.


When you are done you should have the compressed archive with a name something like
\begin{alltt}
rtems-addon-packages-\rtemsPackages.tar.bz2
\end{alltt}

\subsection{Unpack the add-on package sources}
Change to your RTEMS source directory and unpack the RTEMS sources by:
\begin{alltt}
cd /usr/local/source/RTEMS
bzcat rtems-addon-packages-\rtemsPackages.tar.bz2 | tar xf -
\end{alltt}
This will unpack the source for all the RTEMS packages into a directory named
\begin{alltt}
rtems-addon-packages-\rtemsPackages
\end{alltt}


\subsection{Set the RTEMS\_MAKEFILE\_PATH environment variable}
The makefiles in the RTEMS packages use the \verb@RTEMS_MAKEFILE_PATH@ environment variable to determine the
target architecture and board-support package.  For example,
\begin{verbatim}
export RTEMS_MAKEFILE_PATH=/opt/rtems/rtems\rtemsToolVersion/powerpc-rtems\rtemsToolVersion/mvme2100
\end{verbatim}
will select the Motorola Power PC architecture and the RTEMS mvme2100 board-support package.

\subsection{Build and install the add-on packages}
The {\it bit} script in the packages source directory builds and installs all the add-on packages.  To run
the script change directories to the add-on packages directory and execute:
\begin{verbatim}
sh bit
\end{verbatim}

If you are building for more than one architecture or board-support package, you must run the {\it bit} script once
for each variation with \verb@RTEMS_MAKEFILE_PATH@ set to the different architecture and board-support package.

\section{Try running some RTEMS sample applications (optional)}
The RTEMS build process creates some sample applications.  If you're just getting started with RTEMS it's probably
a good idea to verify that you can actually run a simple RTEMS application on your target hardware before trying to
run a full-blown EPICS IOC application.

The actual method of loading an application into a target processor is
hardware-dependent.  Section~\ref{exampleApp} describes a method which
may be used with RTEMS mvme2100 targets.

\chapter{EPICS Base}
The first step in building an EPICS application is to download the EPICS
base source from the APS server and unpack it.   The details on how
to perform these operations are described on the APS web pages and will
not be repeated here.  Make sure you get the R3.14.7 or later release of EPICS.

\section{Specify the location of RTEMS tools and libraries}
You must first let the EPICS Makefiles know where you've installed
the RTEMS development tools and libraries.  The default location is
\begin{verbatim}
/opt/rtems/rtems\rtemsToolVersion
\end{verbatim}
If you've installed the RTEMS tools and libraries in a different location and
ignored the advice to provide a symbolic link from
\begin{verbatim}
/opt/rtems/rtems\rtemsToolVersion
\end{verbatim}
to wherever you've installed RTEMS you need to edit the EPICS configuration
file
\begin{verbatim}
configure/os/CONFIG.Common.RTEMS
\end{verbatim}
In this file you'll find the lines
\begin{verbatim}
RTEMS_BASE=/opt/rtems/rtems\rtemsToolVersion
RTEMS_VERSION=\rtemsToolVersion
\end{verbatim}
while will have to be changed to reflect the directory where you installed RTEMS.

If you didn't install the RTEMS add-on packages you must edit
your EPICS local configuration file (\verb@configure/CONFIG_SITE.Common.RTEMS@)
and change the \verb@EPICSCOMMANDLINE_LIBRARY@ definition to \verb@EPICS@.
You'll also need to add
\begin{verbatim}
OP_SYS_CFLAGS += -DOMIT_NFS
\end{verbatim}
to this file.

\section{Specify the network domain}
If you're not using DHCP/BOOTP to provide network configuration information
to your RTEMS IOCs you should edit
your EPICS local configuration file (\verb@configure/CONFIG_SITE.Common.RTEMS@)
and specify your Internet Domain Name as:
\begin{verbatim}
OP_SYS_CFLAGS += -DRTEMS_NETWORK_CONFIG_DNS_DOMAINNAME=aps.anl.gov
\end{verbatim}

\section{Specify the network interface}
Some RTEMS board support packages support more than one type of network interface.
The pc386 BSP, for example, can be configured to use several different
network interface cards.  By default the EPICS network configuration for
the pc386 BSP loads network drivers for all network interfaces which
support run-time probing so if you've got one of these network interfaces you
don't need to make any changes to the EPICS network configuration.
If not, see the comments in
\begin{alltt}
src/RTEMS/base/rtems_netconfig.c
\end{alltt}
for instructions on selecting a network interface card when building your
EPICS application.

\section{Specify the target architectures}
The \verb@configure/CONFIG_SITE@ file specifies the target
architectures and board support packages to be supported.
For example, I regularly build for a single target:
\begin{verbatim}
CROSS_COMPILER_TARGET_ARCHS=RTEMS-mvme2100
\end{verbatim}
If you want to build for multiple RTEMS targets you would change
this line to something like
\begin{verbatim}
CROSS_COMPILER_TARGET_ARCHS=RTEMS-mvme2100 RTEMS-gen68360 RTEMS-pc386
\end{verbatim}
The format of the target architecture names is {\tt RTEMS-}{\it bspname}, where
{\tt RTEMS-} indicates that the RTEMS development tools and libraries should
be used, and
{\it bspname} is the name of the RTEMS target architecture and
board support package used back in section~\ref{RTEMS_BSP_CONFIG}.

If you build EPICS base on multiple host architectures and want to build
RTEMS target support on only one of those host architectures the target
specification can instead be performed in 
\begin{alltt}
configure/os/CONFIG_SITE.<{\it{hostArch}}>.Common/CONFIG_SITE
\end{alltt}


\section{Build EPICS base}
This step is very simple.  Just change directories to the EPICS \verb@base@ directory
and run
\begin{verbatim}
make
\end{verbatim}
After a while you'll end up with a working set of EPICS base libraries and tools.


\chapter{EPICS Applications}
Now that you've built the EPICS base libraries you're ready to build and run your
first EPICS application.  Once you've got this application running you can
forget about this tutorial and revert to using the standard EPICS documentation.
You can start with your own special application or you can start with the
example application that is provided with the EPICS distribution.
The following sections
describe the procedure to create, build  and run this example application.
\section{The EPICS example application}

\subsection{Build the example application}
\begin{enumerate}

\item Create a new directory to hold the application source and
then `cd' to that directory.

\item Run the \verb@makeBaseApp.pl@ program to create the example application:
\begin{verbatim}
makeBaseApp.pl -t example test
makeBaseApp.pl -i -t example -a RTEMS-mvme2100 test
\end{verbatim}
If you get complaints about not being able to run these commands you've
probably forgotten to put the `bin'
directory created in the previous section on your shell executable search path. 

The `\verb@test@' in the two \verb@makeBaseApp.pl@ commands can be replaced with whatever name you
want to give your example application.
The `\verb@RTEMS-mvme2100@' can be replaced with whatever target architecture you
plan to use to run the example application.

\item Build the example application by running
\begin{verbatim}
make
\end{verbatim}
\end{enumerate}

\subsection{Install the EPICS IOC files on the TFTP/NFS server}
The application build process creates {\tt db} and {\tt dbd} directories
in the top-level application directory and
produces a set of IOC shell commands in
the {\tt st.cmd} file in the {\tt iocBoot/ioctest} directory.  If you chose
an application name different than {\tt test} in the previous step, the
directory name will change accordingly.
These directories and their contents
must be copied to your TFTP/NFS server.  The actual location depends upon
the remote file access technique being used as described
in the following section.


\subsection{Run the example application on an RTEMS IOC}
\label{exampleApp}
Everything's now ready to go.  The only item left is arranging some
way of loading the RTEMS/EPICS application executable image into the IOC.
There are many ways of doing this (floppy disks, PROM images, etc.), but
I find using a BOOTP/DHCP/TFTP
server to be the most convenient.  The remainder of this section describes
how I load executables into my RTEMS-mvme2100 and RTEMS-pc386 IOCs.
If you're running a different type of IOC you'll have to figure out the
required steps on your own.  The RTEMS documentation may provide the
required information since an EPICS IOC application is an RTEMS application
like any other.

Some RTEMS board-support packages require an NTP server on the network.
If such an IOC doesn't receive a time-synchronization packet from an NTP
server the IOC time will be set to 00:00:00, January 1, 2001.

\subsection{Location of EPICS startup script}
If you're using BOOTP/DHCP to provide network configuration information
to your IOC you should use DHCP site-specific option 129 to specify
the path to the IOC startup script.
If you're using PPCBUG you should set the NIOT `Argument File Name' parameter
to the IOC startup script path.

If you're using NFS for remote file access the EPICS initialization uses the
startup script pathname to determine the parameters for the initial NFS mount.
If the startup script pathname begins with a `\verb@/@' the first component
of the pathname is used as both the server path and the local mount point.
If the startup script pathname does not begin with a `\verb@/@' the
first component of the pathname is used as the local mount point and the
server path is ``\verb@/tftpboot/@'' followed by the first component of
the pathname.  This allows the NFS and TFTP clients to have a similar
view of the remote filesystem.

If you're using TFTP for remote file access
the RTEMS startup code
first changes directories to {\tt /epics/}{\it hostname}{\tt /} within
the TFTP server, where {\it hostname} is the Internet host name of the IOC.
The startup code then reads IOC shell commands from the {\tt st.cmd} script
in that directory.
The name ({\tt st.cmd}) and location of the startup script are fixed from the IOCs point of
view so it must be installed in the corresponding location on the TFTP server.
Many sites run the TFTP server with an option which changes its root directory.
On this type of system you'll have to copy the startup script to the
{\tt /epics/}{\it hostname}{\tt /} directory within the TFTP server's root
directory.  On a system whose TFTP server runs with its root
directory set to {\tt /tftpboot} the startup script for the IOC
whose name is {\tt norumx1} would be placed in
\begin{verbatim}
/tftpboot/epics/norumx1/st.cmd
\end{verbatim}
The application build process creates {\tt db} and {\tt dbd} directories
in the top-level application directory.  These directories and their contents
must be copied to the IOC's directory on the TFTP server.  For the 
example described above the
command to install the files for the {\tt norumx1} IOC is
\begin{verbatim}
cp -r db dbd /tftpboot/epics/norumx1
\end{verbatim}

\subsubsection{MVME2100 Using PPCBUG}
\begin{enumerate}
\item Use the PPCBUG ENV command to set the `Network PReP-Boot Mode Enable'
parameter to `Y'.
\item Use the PPCBUG NIOT command to set the network parameters.  Here are the
parameters for a test IOC I use:
\begin{verbatim}
Controller LUN =00
Device LUN     =00
Node Control Memory Address =FFE10000
Client IP Address      =164.54.8.8
Server IP Address      =164.54.8.131
Subnet IP Address Mask =255.255.252.0
Broadcast IP Address   =164.54.11.255
Gateway IP Address     =0.0.0.0
Boot File Name ("NULL" for None)     =/epics/test/bin/RTEMS-mvme2100/example.boot
Argument File Name ("NULL" for None) =/epics/test/iocBoot/iocexample/st.cmd
Boot File Load Address         =001F0000
Boot File Execution Address    =001F0000
Boot File Execution Delay      =00000000
Boot File Length               =00000000
Boot File Byte Offset          =00000000
BOOTP/RARP Request Retry       =00
TFTP/ARP Request Retry         =00
Trace Character Buffer Address =00000000
BOOTP/RARP Request Control: Always/When-Needed (A/W)=W
BOOTP/RARP Reply Update Control: Yes/No (Y/N)       =Y
\end{verbatim}
\item Set up your TFTP/NFS servers.  PPCBUG uses TFTP
to load the executable image then the EPICS initialization uses
NFS to read the EPICS startup script (the `Argument File Name' in the NIOT
parameters).
I set the TFTP server root to /tftpboot and arrange for the NFS server to
export /tftpboot/epics to the IOCs.  This arrangement lets me simply copy the
application tree, beginning at the <top> directory to the TFTP/NFS server area.
\end{enumerate}

\subsubsection{PC386}
\begin{enumerate}
\item Install an EtherBoot bootstrap PROM image obtained from
the `ROM-o-matic' server (http://www.rom-o-matic.net/) on the IOC
network interface cards.
\item Set up your BOOTP/DHCP server to provide the network configuration
parameters to the IOC.
\item The TFTP and NFS servers can be configured as noted above.
\end{enumerate}

The cexp package can be used to
incrementally load your application components.  This package can not
be distributed with RTEMS or EPICS since it uses components covered by
the GNU Public License.


\appendix
\chapter{Script to get and build the cross-development tools}
\label{getAndBuildTools}
If you're feeling brave you can turn loose the following script.  It attempts
to download, unpack, configure, build and install the GNU cross-development
tools and libraries for one or more target architectures.  To use
the script, set the \verb@ARCHS@ 
environment variable to the architectures you wish to support, then 
\begin{verbatim}
sh getAndBuildTools.sh
\end{verbatim}
\verbatimfile{getAndBuildTools.sh}

\end{document}
